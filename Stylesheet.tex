%Präambel (Formatierungsregeln)

\documentclass [12pt, oneside, bibtotoc, liststotoc,]{scrbook}
%\usepackage{bibgerm}       		% deutsche Literaturverzeichnisse
\usepackage[latin1, utf8]{inputenc} 	% Umlaute im Text
\usepackage[T1]{fontenc}
\usepackage[babel,german=quotes]{csquotes}
\usepackage{graphicx} 			% Einfügen von Grafiken
\usepackage{graphics}
\usepackage{wrapfig}
\usepackage{url}          		% URL's (z.B. in Literatur) schöner formatieren
\graphicspath{{./Bilder/}}		% gibt Bilderordner an
\usepackage[ngerman]{babel} 	% neue deutsche Rechtschreibung
\usepackage{mathptmx,courier} 	% setzt die Schriftart auf Times New Roman
\usepackage[scaled]{helvet}		% setzt Schriftart für nicht serife Textteile
\usepackage[a4paper,					% hier werden die Seitenränder eingestellt
left=4.0cm, right=2.0cm,
top=2.5cm, bottom=2.5cm]{geometry}
\usepackage{setspace}			% legt Zeilenabstand fest

%\usepackage{natbib}
%\bibliographystyle{natdin}

\usepackage{chngcntr} 				%Fußnotennummerierung durchgängig
\counterwithout{footnote}{chapter}	%
\usepackage{tocloft}
\setlength{\cftsecnumwidth}{2cm}
%
% der Befehl \hypenation versteht keine Sonderzeichen, also weder �
% noch "a noch \"a. W�rter die derartige Zeichen enthalten m�ssen
% direkt im Text getrennt werden, z.B. W�r\-ter
%
\hyphenation{Ma-nage-ment}
\hyphenation{Ma-nage-ment-agent}
\hyphenation{Ma-nage-ment-agent-en}
\hyphenation{Ma-nage-ment-ar-chi-tek-tur}
\hyphenation{Ma-nage-ment-ar-chi-tek-tu-ren}
\hyphenation{Ma-nage-ment-an-wen-dung}
\hyphenation{Ma-nage-ment-an-wen-dung-en}
\hyphenation{Ma-nage-ment-an-for-der-ung}
\hyphenation{Ma-nage-ment-funk-ti-on}
\hyphenation{Ma-nage-ment-funk-ti-onen}
\hyphenation{Ma-nage-ment-kon-zep-te}
\hyphenation{Ma-nage-ment-res-source}
\hyphenation{Ma-nage-ment-in-for-ma-ti-on}
\hyphenation{Ma-nage-ment-res-sour-cen}
\hyphenation{ma-nage-ment-re-le-vante}
\hyphenation{ma-nage-ment-sy-stem}
\hyphenation{ma-nage-ment-sy-steme}
\hyphenation{Ma-nage-ment-in-stru-men-tie-rung}
\hyphenation{Ma-nage-ment-platt-form}
\hyphenation{Sys-te-men}
\hyphenation{Sys-tem-um-ge-bun-gen}
\hyphenation{Sys-tem-ma-nage-ment}
\hyphenation{DHCP}
\hyphenation{Ma-nage-ment-diszi-plinen}
\hyphenation{System-management-architekturen}
\hyphenation{Verwendungs-nachweise}
\hyphenation{Video-einricht-ungen}
\hyphenation{Res-source}
\hyphenation{Res-sourcen}
\hyphenation{Grund-anwendung}
\hyphenation{Grund-anwendungen}
\hyphenation{Basis-anwendung}
\hyphenation{Core}
\hyphenation{Kom-mu-ni-ka-ti-on}
\hyphenation{De-sign-ent-schei-dung}
\hyphenation{Sprung-ad-res-sen}
\hyphenation{Klas-si-fi-ka-ti-on}
\hyphenation{Schreib-recht}
\hyphenation{Be-nut-zer-zer-ti-fi-kat}
\hyphenation{Bau-stein-ent-wi-ckler}
\hyphenation{ad-mi-ni-stra-ti-ve}

 			% in dieses File kommen Wörter die Latex nicht richtig trennt



\begin{document}
\pagestyle{empty}
% ---------------------------------------------------------------
\frontmatter 					
	\pagestyle{empty}    
    
%
%%%%%%%%%%%%%%%%%%%%%%%%%%%%%%%


% Deckblatt

\thispagestyle{empty}
\begin{flushleft}
Universität Muster\\
MusterFakultät\\
Fachgebiet der Gespenster und Geisteswissenschaften\\
\vspace{0,5cm}
Dozent: Erika Mustermann
\end{flushleft}
\begin{center}
   
    \vspace{7cm}
% Diplomarbeit | Master's Thesis | Bachelorarbeit in Informatik | Wirtschaftsinformatik |
   
% Thema bzw. Titel der Arbeit  (In der Sprache, in der die Arbeit verfasst wurde)
   
    {\Huge \textbf{Lorem Ipsum}}\\ % bei langen Titeln ggf. Schriftgre auf \huge herunter setzen
    \vspace*{0,5cm}
    {\Large \textbf{Untertitel}}\\
    
\end{center}
\vspace{7cm}
\begin{flushleft}


\underline{Verfasser}\\
\vspace{0,5cm}
Name, Vorname\\
Straße xx, 9999 Musterstadt\\
kampfq@posteo.de\\
Matrikelnummer: 3425\\
Studienrichtung: Gespenstes und Geisterwissenschaften

	\vspace{1cm}

Eingereicht am\\

\end{flushleft}
\newpage

%%%%%%%%%%%%%%%%%%%%%%%%%%%%%%%
% Rckseite Deckblatt


%%%%%%%%%%%%%%%%%%%%%%%%%%%%%%%
 
    \thispagestyle{empty}		% ohne Seitenzahl
   
     
    \thispagestyle{empty}		% ohne Seitenzahl
    
    

\chapter*{Vorwort}
\thispagestyle{empty}

\newpage 			% Abstract
    \thispagestyle{empty}		% ohne Seitenzahl
	
    
    \tableofcontents 			% Inhaltsverzeichnis
	\thispagestyle{empty}		% ohne Seitenzahl
	
% ---------------------------------------------------------------
\mainmatter 					% die eigentliche Arbeit
	
	\onehalfspacing				% Zeilen-> abstand 1,5
	\pagestyle{plain}			% ohne Kopfzeile
	
	
    \chapter{Kapitel 1}

\endinput
    
    

% ---------------------------------------------------------------
\backmatter 					% ab hier keine Nummerierung mehr
    
    \pagenumbering{Roman}		% Römische Seitenzahlen
    
	\renewcommand{\thesection}{Anhang \arabic{section}:}
	
   	\chapter{Anhang}

\endinput
   	\chapter{Quellen}

\endinput
   	
\chapter{Begriffsverzeichnis}

\textbf{Hallo} engl.: "hello"; Hallo ist
\begin{itemize}
\item einfach Hallo


\end{itemize}

\endinput
   	
   	
    % deutsche Bibliographie
    \bibliography{./Bib/bib}	% die eigendliche Bibliographie
	\listoffigures				% Abbildungsverzeichnis
\end{document}
